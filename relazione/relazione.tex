\documentclass[11pt, a4paper, titlepage, block]{article}
\usepackage{listings}
\usepackage{graphicx}
\hyphenpenalty=10000

\begin{document}
	\begin{titlepage}
		
		\newcommand{\HRule}{\rule{\linewidth}{0.5mm}} % Defines a new command for the horizontal lines, change thickness here
		
		\center % Center everything on the page
		
		%----------------------------------------------------------------------------------------
		%  HEADING SECTIONS
		%----------------------------------------------------------------------------------------
		
		\textsc{\LARGE Universit\`a di Urbino}\\[1.5cm] % Name of your university/college
		\textsc{\Large Informatica Applicata}\\[0.5cm] % Major heading such as course name
		\textsc{\large Programmazione Procedurale e Logica}\\[0.5cm] % Minor heading such as course title
		
		%----------------------------------------------------------------------------------------
		%  TITLE SECTION
		%----------------------------------------------------------------------------------------
		
		
		\HRule \\[0.4cm]
		{ \huge \bfseries Relazione}\\[0.2cm] % Title of your document
		\HRule \\[0.4cm]
		\textsc{\large Progetto per la sessione invernale 2014/2015}
		\\[2cm]
		%----------------------------------------------------------------------------------------
		%  AUTHOR SECTION
		%----------------------------------------------------------------------------------------
		
		\begin{minipage}{\textwidth}
			\begin{flushleft}
				\emph{Studente:}\\
				Marco \textsc{Tamagno}\\ % Your name
				matricola no: 261985
				\\[1cm]
				\emph{Studente:}\\
				Francesco \textsc{Belacca}\\ % Your name
				matricola no: 260492\\
			\end{flushleft}
		\end{minipage}\\[3cm]
		
		\begin{minipage}{\textwidth}
			\begin{flushright}
				\emph{Professore:} \\
				Marco \textsc{Bernardo}\\ % Supervisor`s Name
			\end{flushright}
		\end{minipage}\\[4cm]

		{\today}\\[1cm]


		%----------------------------------------------------------------------------------------
		%  DATE SECTION
		%----------------------------------------------------------------------------------------
		
	 % Date, change the \today to a set date if you want to be precise		
		%----------------------------------------------------------------------------------------
		%  LOGO SECTION
		%----------------------------------------------------------------------------------------		
		%\includegraphics{Logo}\\[1cm] % Include a department/university logo - this will require the graphicx package		
		%----------------------------------------------------------------------------------------	
		\newpage
		\tableofcontents
		\newpage
		
	\end{titlepage}
	
	\section{Specifica del Problema}
	Write an ANSI C library that manages binary relations by exporting the following functions. The first C function returns a binary relation introduced through the keyboard. The second C function has a binary relation as input parameter and prints it to the screen. The third C function has a binary relation as input parameter and establishes whether it is a partial order relation, printing to the screen which property does not hold in the case that the relation is not such. The fourth C function has a binary relation as input parameter and establishes whether it is a total order relation, printing to the screen which property does not hold in the case that the relation is not such. The fifth C function has a binary relation as input parameter and establishes whether it is an equivalence relation, printing to the screen which property does not hold in the case that the relation is not such. The sixth C function has a binary relation as input parameter and establishes whether it is a mathematical function; if it is not, then the element violating the property will be printed to the screen, otherwise a message will be printed to the screen indicating whether the function is injective, surjective, or bijective.
	[The project can be submitted also by first-year students.]\\
	\\
	Scrivere una libreria ANSI C che gestisce le relazioni binarie esportando le seguenti funzioni. La prima funzione C restituisce una relazione binaria acquisita da tastiera. La seconda funzione C ha come parametro di ingresso una relazione binaria e la stampa a video. La terza funzione C ha come parametro di ingresso una relazione binaria e stabilisce se essa \`e una relazione d'ordine parziale, stampando a video quale propriet\`a non vale nel caso la relazione non sia tale. La quarta funzione C ha come parametro di ingresso una relazione binaria e stabilisce se essa \`e una relazione d'ordine totale, stampando a video quale propriet\`a non vale nel caso la relazione non sia tale. La quinta funzione C ha come parametro di ingresso una relazione binaria e stabilisce se essa \`e una relazione d'equivalenza, stampando a video quale propriet\`a non vale nel caso la relazione non sia tale. La sesta funzione C ha come parametro di ingresso una relazione binaria e stabilisce se essa \`e una funzione matematica; se non lo \`e, allora si stamper\`a a video quale elemento violi la propriet\`a, altrimenti si stamper\`a a video un messaggio che indica se la funzione \`e iniettiva, suriettiva o biiettiva.
	[Il progetto pu\`o essere consegnato anche da studenti del primo anno.]\\
	\newpage
	\section{Analisi del Problema}
	\subsection{Input}
	
	
	1. Per l'acquisizione come input abbiamo una relazione binaria del tipo { (a,b); (a1,b1); (a2,b2); ...} formata da un numero non precedentemente definito di coppie che viene acquisita da tastiera;\\
	2. Come input per le altre 5 funzioni abbiamo una relazione binaria.\\
	\subsection{Output}
	
	
	1. Il primo problema (problema dell'acquisizione) restituisce una relazione binaria del tipo { (a,b); (a1,b1); (a2,b2); ...} formata da un numero non precedentemente definito.\\
	\\
	2. Il secondo problema (problema della stampa) stampa a video la relazione binaria che viene dato in pasto alla funzione;\\
	\\
	3. Il terzo problema (problema della verifica dell ordine parziale) ci richiede di controllare se la relazione binaria data in pasto alla funzione \`e una relazione d'ordine parziale, e nel caso in cui non lo sia di andare a stampare a video le varie propriet\`a che non rispetta;\\
	\\
	4. Il quarto problema (problema della verifica dell ordine totale) ci richiede di controllare se la relazione binaria data in pasto al programma \`e una relazione d'ordine totale, e nel caso in cui non lo sia di andare a stampare a video le varie propriet\`a che non rispetta;\\
	\\
	5. Il quinto problema (problema della verifica dell'ordine di equivalenza) ci richiede di controllare se la relazione binaria data in pasto al programma \`e una relazione di equivalenza, e nel caso in cui non lo sia di andare a stampare a video le varie propriet\`a che non rispetta;\\
	\\
	6. Il sesto problema (problema della verifica della funzione) ci richiede di controllare se la relazione binaria data in pasto al programma \`e una funzione, e nel caso in cui non lo sia di andare a stampare a video le varie propriet\`a che non rispetta, mentre nel caso in cui sia una funzione di controllare se tale funzione rispetti le propiet\`a di suriettivit\`a e iniettivit\`a, stampando a video se la funzione \`e suriettiva, iniettiva o biiettiva;\\
	
	
	
	\newpage   
\section{Progettazione dell'Algoritmo}
\subsection{Teoria}
Per lo sviluppo di questo programma si necessita di alcuni cenni di Teoria degli insiemi quali:\\
\\
Concetto di Relazione Binaria: una relazione binaria \`e un sottoinsieme del prodotto
cartesiano di due insiemi (i quali potrebbero pure coincidere,
ma ci\`o non \`e garantito) .\\
\\
Concetto di Relazione d'Ordine Parziale: In matematica, pi\`u precisamente in teoria degli ordini, una relazione d'ordine o ordine su di un insieme \`e una relazione binaria tra elementi appartenenti all'insieme che gode delle seguenti propriet\`a:\\
* riflessiva\\
* antisimmetrica\\
* transitiva.\\
\\
Concetto di Relazione d'Ordine Totale: Una relazione d'ordine si dice Totale, quando oltre a essere parziale soddisfa anche la propiet\`a di Dicotomia (tutti gli elementi devono essere in relazione con ogni altro elemento presente)\\
\\
Concetto di riflessivit\`a: In logica e in matematica, una relazione binaria R in un insieme X \`e detta riflessiva se ogni elemento di X \`e in tale relazione con se stesso. \\
\\
Concetto di transitivit\`a: In matematica, una relazione binaria R in un insieme X \`e transitiva se e solo se per ogni a, b, c appartenenti ad X, se a \`e in relazione con b e b \`e in relazione con c, allora a \`e in relazione con c. \\
\\
Concetto di simmetricit\`a: In matematica, una relazione binaria R in un insieme X \`e simmetrica se e solo se, presi due elementi qualsiasi a e b, vale che se a \`e in relazione con b allora anche b \`e in relazione con a.\\
\\
Un sottoinsieme f di A x B \`e una funzione se ad ogni
elemento di A viene associato da f al pi\`u un elemento di B,
dando luogo alla distinzione tra funzioni totali e parziali
 (a seconda che tutti o solo alcuni degli elementi di A
abbiano un corrispondente in B) e lasciando non specificato
se tutti gli elementi di B siano i corrispondenti di
qualche elemento di A oppure no.\\
\newpage

Concetto di Iniettivit\`a: ad ogni elemento del codominio corrisponde al pi\`u
un elemento del dominio, cio\`e elementi diversi del dominio
vengono trasformati in elementi diversi del codominio.\\
\\
Concetto di Suriettivit\`a: Una funzione si dice suriettiva quando ogni elemento del codominio viene raggiunto da un elemento del dominio.\\
\\
\newpage
\subsection{Scelte di Progetto}

- Una relazione binaria prende in considerazione due elementi, questi due elementi si potrebbero vedere come due variabili distinte che poi andranno a far parte della stessa struttura, per questo riteniamo opportuno creare una struttura dati che inglobi entrambi gli elementi.\\
\\
- I due termini potrebbero essere numerici, ma non \`e detto, quindi per completezza riteniamo opportuno far scegliere all'utente se inserire elementi di tipo numerico, o altro (simboli,lettere etc.) a seconda delle sue necessit\`a.\\
\\
- A priori, prendendo come input una relazione binaria, non possiamo sapere se tutti gli elementi del primo insieme sono in relazione con almeno un elemento del secondo insieme o se tutti gli elementi del secondo insieme fanno parte di una coppia ordinata, quindi \`e opportuno chiedere all'utente se ci sono elementi isolati che non fanno parte di nessuna coppia ordinata.\\ 
\newpage
Breve lista delle funzioni da utilizzare:\\
\\
	\subsection{Funzioni per l`acquisizione}
	
	acquisizione: per acquisire la relazione.\\
	\\
	
	\subsection{Funzioni per la verifica delle propriet\`a:}

	controllo\textunderscore iniettivit\`a: serve a controllare se l'iniettivit\`a \`e rispettata o meno.\\
	\\
	controllo\textunderscore transitivit\`a: serve a controllare se la transitivit\`a 
	viene rispettata o meno.\\
	\\
	controllo\textunderscore antisimmetria: serve a controllare se l'antisimmetria viene rispettata o meno.\\
	\\
	controllo\textunderscore simmetria: serve a controllare se la simmetria viene rispettata o meno.\\
	\\
	controllo\textunderscore riflessivit\`a: serve a controllare se la riflessivit\`a viene rispettata o meno.\\
	\\
	controllo\textunderscore dicotomia: serve a verificare se la dicotomia viene rispettata o meno.\\
	\\
	controllo\textunderscore suriettivit\`a: serve a verificare se la suriettivit\`a viene rispettata o meno.\\
	\\
	\newpage
	\subsection{Funzioni principali:}
	ordine\textunderscore parziale: richiama le funzioni delle propriet\`a e controlla se c'\`e un ordine parziale (stampa a video se c'\`e o meno un ordine parziale, e nel caso non c'\`e stampa quali propriet\`a non vengono rispettate) .\\
	\\
	ordine\textunderscore totale: richiama la funzione ordine\textunderscore parziale e controllo\textunderscore dicotomia e controlla se c'\`e un ordine totale (stampa a video se esiste o meno un ordine totale, e nel caso non c'\`e stampa quali propiet\`a non vengono rispettate) .\\
	\\
	relazione\textunderscore equivalenza: richiama le funzioni delle propriet\`a e controlla se c'\`e una relazione d'equivalenza (stampa a video se c'\`e o meno una relazione d'equivalenza, e nel caso non c'\`e stampa a schermo quali propriet\`a non vengono rispettate) .\\
	\\
	controllo\textunderscore funzione: verifica se la relazione \`e una funzione (stampa a video se c'\`e o non c'\`e una funzione e nel caso non ci sia dice quale coppia non soddisfa le propriet\`a) .\\
	\\
	\newpage 
	\subsection{Input}
	
	
	Per l'input abbiamo necessit\`a di usare una struttura dati dinamica, nella quale andiamo a salvare la Relazione Binaria dataci dall'utente, il numero delle coppie e il tipo di input (numerico o per stringhe) .\\
	\\
	L'input dovr\`a essere dotato di diversi controlli, se l'utente sceglie di inserire un input di tipo numerico allora non potr\`a digitare stringhe e/o caratteri speciali etc.\\
	\\
	La scelta di due tipi di input differente dovr\`a essere data per dare la possibilit\`a all'utente nel caso scelga di fare un'input di tipo numerico di poter effettuare operazioni non legate alle funzioni della libreria, (esempio: l'utente vuole decidere di moltiplicare l'input per due, e vedere se mantiene le propiet\`a, con un'input di tipo numerico l'utente pu\`o farlo e ci\`o avrebbe un senso, con un'input di tipo stringa meno) .\\
	\\
	La scelta dell'input di tipo stringa dovr\`a essere data per aver maggior completezza, una relazione binaria non deve essere forzatamente numerica ma pu\`o essere anche tra cose, oggetti, animali, colori e qualsiasi altra cosa possa venire in mente.\\
	\\
	Alle varie funzioni verr\`a data come input la struttura dati salvata in precedenza dalla funzione Acquisizione, per poterne verificare le varie propiet\`a.\\
	
	
	
	
	\newpage 
	\subsection{Output - Acquisizione}
	Durante l'acquisizione avremo diversi output video che guideranno l'utente nell'inserimento dei dati, e che segnaleranno eventuali errori commessi.
	Finita l'acquisizione dovremo restituire l'indirizzo della struttura, che all'interno quindi conterr\`a i dati inseriti dall'utente. Abbiamo scelto di fare ci\`o perch\`e non essendo permesso l'utilizzo di variabili globali, il modo pi\`u semplice di passare i dati inseriti da una funzione all'altra \`e quello di creare una struttura dinamica.
	Una volta restituito l'indirizzo della struttura, a seconda della funzione lanciata nel file Test.c si lanceranno le altre 5 funzioni, dato che queste prendono tutte in pasto l'output della prima (cio\`e l'indirizzo della struttura della relazione binaria) e la utilizzano per verificarne varie propriet\`a.\\
	\\
	\subsection{Output - stampa}
	La funzione “stampa” avr\`a come output la stampa a video della struttura acquisita, con qualche aggiunta grafica (le parentesi e le virgole) per rendere il tutto pi\`u facilmente interpretabile e leggibile.\\
	\\
	\subsection{Output - ordine\textunderscore parziale}
	La funzione “ordine\textunderscore parziale” avr\`a come output la stampa a video del risultato della verifica delle propriet\`a di riflessivit\`a antisimmetria e transitivit\`a. Nel caso in cui siano tutte verificate si stamper\`a che la relazione \`e una relazione di ordine parziale, mentre nel caso in cui non siano verificate si stamper\`a che non lo \`e e il perch\`e (cio\`e quale (o quali) propriet\`a non \`e verificata (o non sono verificate) .\\
	\\
	\subsection{Output - ordine\textunderscore totale}
	La funzione “ordine\textunderscore totale” avr\`a come output la stampa a video del risultato della verifica delle propriet\`a necessarie ad avere una relazione d'ordine parziale, e verificher\`a poi se anche la dicotomia \`e valida per la relazione o meno. Nel caso in cui tutto sia positivo, allora si stamper\`a che la relazione \`e di ordine totale, mentre se non lo \`e si stamper\`a cosa fa in modo che non lo sia.\\
	\\
	\subsection{Output - relazione\textunderscore equivalenza}
	La funzione “relazione\textunderscore equivalenza” avr\`a come output la stampa a video del risultato della verifica delle propriet\`a di riflessivit\`a simmetria e transitivit\`a e nel caso in cui siano tutte positive si stamper\`a che la relazione \`e una relazione di equivalenza, mentre nel caso in cui qualcosa non sia verificato si stamper\`a ci\`o che impedisce alla relazione di essere una relazione d'equivalenza.\\
	\\
	\subsection{Output - controllo\textunderscore funzione}
	La funzione ”controllo\textunderscore funzione” avr\`a come output la stampa a video della verifica della propriet\`a che rende la relazione binaria una funzione, e in caso lo sia,se questa \`e sia suriettiva e iniettiva, e in caso sia entrambe si stamper\`a che la relazione binaria oltre ad essere una funzione \`e una funzione biiettiva.\\
	\\
	\newpage
	\section{Implementazione dell'Algoritmo}
	\subsection{Libreria (file .h) }
	\lstset{numbers=left, tabsize=2,breaklines=true, language=C}
	\lstinputlisting{../librerie/lib_rel_bin.h} 
	\newpage
	\subsection{Libreria (file .c) }
	\lstset{numbers=left, tabsize=2,breaklines=true, language=C}
	\lstinputlisting{../librerie/lib_rel_bin.c} 
	\newpage	
	\subsection{Test}
	\lstset{numbers=left, tabsize=2,breaklines=true, language=C}
	\lstinputlisting{../Test.c}
	\newpage
\subsection{Makefile}
\begin{tabbing}
	Test.exe: \=Test.c Makefile\\
	\>gcc -ansi -Wall -O Test.c lib\textunderscore rel\textunderscore bin.c -o Test.exe\\
	pulisci:\\
	\>rm -f Test.o\\
	pulisci\textunderscore tutto:\\
	\>rm -f Test.exe Test.o\\
\end{tabbing}
	
	
	
	
	
	
	
	\newpage
	\section{Testing del programma}
	\subsection{Test 1:}
	Test di Relazione d'ordine Totale.\\
	\\
	Inputs: (a,a) (a,b) (b,b) \\
	\\
	Outputs: controlloriflessivit\`a: 1,controllosimmetria: 0, controllotransitivit\`a: 1
	controllodicotomia: 1, la relazione \`e una relazione d'ordine totale in quanto \`e rispetta anche la propiet\`a di Dicotomia.\\
	\includegraphics[width=3in,height=3in,viewport=0 0 300 300]{../Screenshots/Test1Input.jpg}
	\\
	\includegraphics[width=3in,height=3in,viewport=0 0 300 300]{../Screenshots/Test1Output.png}
	\newpage
	\subsection{Test 2:}
	Test di Relazione d'ordine Parziale.\\
	\\
	Inputs: (a,a) (b,b) (a,b) (c,c) \\
	\\
	Outputs:controlloriflessivit\`a: 1,controllosimmetria: 0, controllotransitivit\`a: 1
	la relazione \`e una relazione d'ordine parziale in quanto rispetta le propriet\`a.\\
	\includegraphics[width=3in,height=3in,viewport=0 0 300 300]{../Screenshots/Test2Input.png}
	\\
	\includegraphics[width=3in,height=3in,viewport=0 0 300 300]{../Screenshots/Test2Output.png}
	\\
	\\
	\newpage
	\subsection{Test 3:} 
	Test di Relazione d'ordine non Parziale.\\
	\\
	Inputs: (a,a) (b,b) (c,c) (d,d) (e,e) (a,b) (b,c) \\
	\\
	Outputs:controlloriflessivit\`a: 1,controllosimmetria: 0, controllotransitivit\`a: 0
	la relazione non \`e una relazione d'ordine parziale in quanto non rispetta le propriet\`a.\\
	\includegraphics[width=3in,height=3in,viewport=0 0 300 300]{../Screenshots/Test3Input.png}
	\\
	\includegraphics[width=3in,height=3in,viewport=0 0 300 300]{../Screenshots/Test3Output.png}
	\\
	\\
	\newpage
	\subsection{Test 4:}
	Test di Relazione d'equivalenza.\\
	\\
	Inputs: (a,a) (a,b) (b,a) (b,b) \\
	\\
	Outputs:controlloriflessivit\`a: 1,controllosimmetria: 1, controllotransitivit\`a: 1
	controllodicotomia: 0, la relazione \`e una relazione d'equivalenza in quanto rispetta le propriet\`a.\\
	\includegraphics[width=3in,height=3in,viewport=0 0 300 300]{../Screenshots/Test4Input.png}
	\\
	\includegraphics[width=3in,height=3in,viewport=0 0 300 300]{../Screenshots/Test4Output.png}
	\\
	\\
	\newpage
	\subsection{Test 5:}
	Test di Relazione non d'equivalenza.\\
	\\
	Inputs: (a,a) (a,b) (b,c) \\
	Outputs:controlloriflessivit\`a: 0,controllosimmetria: 0, controllotransitivit\`a: 0
	la relazione non \`e una relazione d'ordine d'equivalenza in quanto non rispetta le propriet\`a.\\
	\includegraphics[width=3in,height=3in,viewport=0 0 300 300]{../Screenshots/Test5Input.png}
	\\
	\includegraphics[width=3in,height=3in,viewport=0 0 300 300]{../Screenshots/Test5Output.png}
	\\
	\\
	\newpage
	\subsection{Test 6:}
	Test di Funzione.\\
	\\
	Inputs: (a,a) 
	Outputs:La relazione binaria \`e una funzione.\\
	La relazione binaria \`e iniettiva.\\
	La relazione binaria \`e biiettiva.\\
	\includegraphics[width=3in,height=3in,viewport=0 0 300 300]{../Screenshots/Test6Input.png}
	\\
	\includegraphics[width=3in,height=3in,viewport=0 0 300 300]{../Screenshots/Test6Output.png}
	\\
	\\
	\newpage
	\subsection{Test 7:}
	Test per verificare il controllo degli inputs.\\
	\\
	Inputs: (casa rossa,casa blu) (casa blu,casa blu) (casa rossa,casa rossa) \\
	\\
	Outputs:controllo\textunderscore riflessivit\`a: 1,controllo\textunderscore simmetria: 1, controllo\textunderscore transitivit\`a: 1 dicotomia: 1
	la relazione \`e una relazione d'ordine totale in quanto rispetta le propriet\`a.\\
	le funzioni funzionano anche con input contenti degli spazi.\\
	\includegraphics[width=3in,height=3in,viewport=0 0 300 300]{../Screenshots/Test7Input.png}
	\\
	\includegraphics[width=3in,height=3in,viewport=0 0 300 300]{../Screenshots/Test7Output.png}
	\\
	\\
	\newpage
	\subsection{Test 8:}
	Test per inserire stringhe in una relazione numerica.\\
	\\
	Inputs: (1,a) \\
	\\
	Outputs: c'\`e un errore reinserisci il valore.\\
	\\
	stampa errore in quanto si era selezionato di voler immettere un input di tipo numerico.\\
	\includegraphics[width=3in,height=3in,viewport=0 0 300 300]{../Screenshots/Test8Input.png}
	\\
	\\
	\newpage
	\subsection{Test 9:}
	Test per vedere se una relazione binaria qualunque e'una funzione.\\
	Inputs: (1,2) (1,1) \\
	\\
	Outputs: La relazione binaria non \`e una funzione\\
	Nel 2 elemento c'\`e un errore che impedisce alla relazione binaria di essere una funzione;\\
	\includegraphics[width=3in,height=3in,viewport=0 0 300 300]{../Screenshots/Test9Input.png}
	\\
	\includegraphics[width=3in,height=3in,viewport=0 0 300 300]{../Screenshots/Test9Output.PNG}
	\\
	\\
	\newpage
	\subsection{Test 10:}
	Inputs: (1,1) (2,1) \\
	\\
	Outputs: La relazione binaria \`e una funzione\\
	Nel 2 elemento c'\`e un errore che impedisce alla funzione di essere iniettiva\\
	La funzione non \`e iniettiva\\
	La funzione non \`e biiettiva\\
	\includegraphics[width=3in,height=3in,viewport=0 0 300 300]{../Screenshots/Test10Input.png}
	\\
	\includegraphics[width=3in,height=3in,viewport=0 0 300 300]{../Screenshots/Test10Output.PNG}
	\newpage
	\section{Verica del programma}

Questa porzione di codice fa in modo che una volta eseguito si abbia nel valore c la sommatoria del numero di elementi distinti inseriti dall'utente.	
\\
\\
riscontro = numero\textunderscore elementi\\	
while (numero\textunderscore elementi\textgreater 0) \\
\{
numero\textunderscore elementi - -;\\
riscontro = riscontro + numero\textunderscore elementi;\\
\}
\\
\\
La postcondizione \`e 
\\
\\
\\
R = (riscontro = $\displaystyle \sum_{j=0}^{numero\textunderscore elementi-1} numero\textunderscore elementi - j $\\
\\
\\
\\
si pu\`o rendere la tripla vera mettendo precondizione vero in quanto:\\
\\
\\
\\
 -Il predicato
 
 $P = (numero\textunderscore elementi \textgreater 0 \wedge riscontro = \displaystyle \sum_{j=0}^{numero\textunderscore elementi-1} numero\textunderscore elementi - j $) \\
 \\ 
 \\
 \\
 e la funzione: 
 \\
 \\
 \\
 tr (numero\textunderscore elementi) = numero\textunderscore elementi - 1) 
 \\
 \\
 \\
 soddisfano le ipotesi del teorema dell'invariante di ciclo in quanto:\\
 \\
 \\
 \\
 $\ast \lbrace P \wedge numero\textunderscore elementi\textgreater 0 \rbrace riscontro = riscontro + numero\textunderscore elementi; numero\textunderscore elementi = numero\textunderscore elementi - -;\lbrace P \rbrace $ \\
 \\
 \\
 \\
 segue da: \\
 \\
 \\
 \\
 $P_{numero\textunderscore elementi,numero\textunderscore elementi -1} \wedge riscontro \displaystyle \sum_{j=0}^{numero\textunderscore elementi-2} numero\textunderscore elementi - j $
 \\
 \\
 \\
 e donatoto con P'quest'ultimo predicato, da:
 \\
 \\
 \\
 P'$_{riscontro,riscontro + numero\textunderscore elementi} = (numero\textunderscore elementi \textgreater 0 \wedge riscontro + numero\textunderscore elementi $ = \\\\ =$ \displaystyle \sum_{j=0}^{numero\textunderscore elementi-2} numero\textunderscore elementi - j $) \\
 \\
 \\
 
 P'$_{riscontro,riscontro + numero\textunderscore elementi} = (numero\textunderscore elementi \textgreater 0 \wedge c$ =\\\\=$ \displaystyle \sum_{j=0}^{numero\textunderscore elementi-1} numero\textunderscore elementi - j $) \\
 \\
 \\
 
 
 in quanto denotato con P$''$ quest'ultimo predicato, si ha:
 (P $\wedge$ numero$\textunderscore elementi \textgreater 1) = (numero \textunderscore elementi \textgreater 0 \wedge riscontro = \displaystyle \sum_{j=0}^{numero\textunderscore elementi-1} numero\textunderscore elementi - j \wedge numero\textunderscore elementi \textgreater 1$) \\
 $|= P''$\\
 $\ast$ Il progresso \`e garantito dal fatto che tr (numero\textunderscore elemnti) decresce di un unit\`a ad ogni iterazione in quanto numero\textunderscore elementi viene decrementata di un'unit\`a ad ogni iterazione.\\
 $\ast$ La limitatezza segue da:\\
 (P$\wedge$$tr (numero\textunderscore elementi) \textless 1) = (numero\textunderscore elementi \textgreater 0 \wedge c = \displaystyle \sum_{j=0}^{numero\textunderscore elementi-1} numero\textunderscore elementi - j \wedge numero\textunderscore elementi \textgreater 1$) //
 \\
 $\equiv (riscontro = \displaystyle \sum_{j=0}^{numero\textunderscore elementi-1}$numero\textunderscore elementi -j) \\
\\
\\
\\
$|=$ numero\textunderscore elementi \textgreater numero\textunderscore elementi - 1

Poich\`e: \\
\\
\\
$ (P \wedge numero\textunderscore elementi \textless 1 ) = (numero\textunderscore elementi \textgreater 0 \wedge riscontro= (P\wedge numero\textunderscore elementi \textgreater 1) = ( numero\textunderscore elementi \textgreater 0 \wedge riscontro =$\\\\$= \displaystyle \sum_{j=0}^{numero\textunderscore elementi-1} numero\textunderscore elementi - j $ $\wedge numero\textunderscore elementi \textless 1) $ \\\\$
\equiv (numero\textunderscore elementi = 1 \wedge riscontro =\\\\= \displaystyle \sum_{j=0}^{numero\textunderscore elementi-1} numero\textunderscore elementi - j \wedge numero\textunderscore elementi \textless 1) ) $) 
\\
\\
\\

Dal corollario del teorema dell invariabilit\'a di ciclo si ha che P pu\`o essere usato solo come precondizione dell'intera istruzione di ripetizione.
\\
\\
-Proseguendo infine a ritroso si ottiene prima:

 $P_{numero\textunderscore elementi,0} = ( 0\textless= 0\textless= numero\textunderscore elementi \wedge riscontro = \displaystyle \sum_{j=0}^{0-1} numero\textunderscore elementi - j$) (riscontro = 0) 
 \\
 \\
 \\
 e poi, denotato con P$'''$ quest'ultimo predicato si ha:
 \\
 \\
 \\
 P$'''$$_{riscontro,0}$ = (0 = 0) = vero

\end{document}

